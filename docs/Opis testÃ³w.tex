% Ten plik jest zaprojektowany do użycia z LuaLaTeX.

\documentclass[a4paper,12pt]{article}
\usepackage{fontspec} % Obsługa czcionek w LuaLaTeX
\usepackage[polish]{babel}
\usepackage{geometry} % Ustawienia marginesów
\geometry{margin=1in}
\usepackage{hyperref} % Obsługa hiperłączy
\usepackage{titlesec} % Formatowanie sekcji
\usepackage[skip=10pt plus1pt]{parskip}
\usepackage[table]{xcolor}
\usepackage{float}

% Ustawienia hiperłączy
\hypersetup{
    colorlinks=true,
    linkcolor=black,
    urlcolor=blue,
    pdftitle={Opis testów},
    pdfauthor={Diego Ostoja Kowalski}
}

% Tytuł dokumentu
\title{
    \line(1,0){250}\\
    System Zarządzania Promocjami\\
    Opis testów\\
    \line(1,0){250}}
\author{Antoni Blicharz\\
        Szymon Dybał\\
        Jakub Koszorz\\
        Mikołaj Mroczek\\
        Diego Ostoja-Kowalski\\}
\date{\today}

\begin{document}

\begin{titlepage}
    \maketitle
\end{titlepage}

\newpage

\section{Opis testowania}

\subsection{Wstep}

Proces testowania aplikacji został przeprowadzony w celu zapewnienia najwyższej jakości produktu finalnego oraz spełnienia wymagań funkcjonalnych i niefunkcjonalnych klienta.

\subsection{Przebieg testowania}

Testowanie odbywało się w dwóch częściach.
Testowanie było implementowane równolegle do tworzenia aplikacji, przez osoby odpowiadające jej tworzeniu.
Miało miejsce również testowanie zwieńczające, wykonywane w końcowym etapie tworzenia projektu.
Testowaniem końcowym zajmował się tester, który nie miał tak dużej wiedzy na temat wewnętrznego działania aplikacji, co programiści.
To dało mu możliwość podejścia do aplikacji bez stronniczego patrzenia na nią.
Mógł się wcielić w rolę użytkownika, widzącego aplikacje po raz pierwszy.
Oczywiście tester miał dostęp do kodu aplikacji, więc przy mniejszych błędach poprawiał je własnoręcznie, lecz przy większych błędach strukturalnych zostały one zgłaszane do programistów, którzy je poprawiali.

\subsection{Rodzaje testowania}

Testowanie wykonywane przez testera pokrywało zarówno funkcjonalne, jak i niefunkcjonalne wymagania aplikacji.
Format plików zawierających informacje o znalezionych błędach był następujący:
\begin{itemize}
    \item Pierwsza część zawierała informacje o błędach niebędącymi czymś, co nie mogłoby jednoznacznie i obiektywnie być uznane za błąd, ale co znacznie negatywnie wpływało na doświadczenie używania aplikacji.
    \item W drugiej części opisywane były błędy, które jednoznacznie wskazywały na niepoprawnie działające elementy aplikacji.
\end{itemize}

Testowanie jednostkowe – tester używał aplikacji z perspektywy użytkowników.
Szukanie błędów było przeprowadzane za pomocą zwykłego korzystania z aplikacji oraz przy użyciu narzędzia TestFX.
Sprawdzone były przypadki brzegowe we wszystkich czynnościach, które mógł wykonać użytkownik.

\subsection{Implementowane techniki i narzędzia}


Do testów wykorzystano narzędzie TestFX, umożliwiające automatyzację testów interfejsu użytkownika w aplikacjach JavaFX.
Dodatkowo przeprowadzono testy manualne, weryfikujące poprawność działania funkcjonalności oraz reakcję aplikacji na niestandardowe dane wejściowe.
W celu śledzenia błędów zastosowano prosty system raportowania w formie plików tekstowych, zawierających opis problemów i ich priorytet.


\section{Pierwsza faza testów}

Sekcja pierwsza:
\begin{itemize}
    \item W administratorze, w Report Issues po wprowadzeniu potwierdzeniu z wprowadzeniem złej daty zerowane są wszystkie wpisane pola, zamiast tylko pola daty.
    \item Podanie złej ilości nie wyświetla komunikatu tak jak na przykład podanie złej daty -- przyciski wysłania po prostu nie działa.
\end{itemize}

Sekcja druga:
\begin{itemize}
    \item W administratorze, w Report Issues nie jest odrzucone zgłoszenie błędu w przypadku podania złej godziny w dacie, np. przechodzi zarówno -1.-1.-1 jak i 30.30.30 (powinno być od 0.0.0 do 23.59.59).
    \item Podczas działania programu na moim komputerze ciągle w terminalu Javy samoistnie wyświetało się "Błąd podczas dodawania danych: Cannot add or update a child row: a foreign key constraint fails (\`{}io\textunderscore{}baza\`{}.\`{}RewardsToIssuance\`{}, CONSTRAINT \`{}RewardsToIssuance\textunderscore{}ibfk\textunderscore{}1\`{} FOREIGN KEY (\`{}IssuanceID\`{}) REFERENCES \`{}RewardIssuance\`{} (\`{}IssuanceID\`{}))".
    \item Panele analityka i promotora nie działają wcale.
\end{itemize}

\section{Druga faza testów}

Sekcja pierwsza:
\begin{itemize}
    \item Przy zgłaszaniu defektu w panelu pracownika punktu sprzedaży i tworzeniu i modyfikowaniu nowego defektu po zakończeniu operacji powodzeniem powinno się wracać automatycznie do menu.
\end{itemize}

Sekcja druga:
\begin{itemize}
    \item Uruchomienie aplikacji przez plik \texttt{startIO.bat} nie działa -- wyświetla się okienko logowania, a po zalogowaniu się nic się nie dzieje, trzeba odpalać przez \texttt{javafx:run}.
    \item Przycisk powrotny (przycisk menu po lewej) w promotorze nie działa, a przez to, że nie ma specjalnego przycisku ,,powrót'' jak w operacjach pracownika punktu sprzedaży, a po udanej operacji nie powraca się automatycznie do menu, utykamy na tym panelu i trzeba się wylogować.
    \item W sekcji logistyka patrzenie na zapytania na stacji centralnej w Warszawie nadal prowadzi do pustej strony.
    \item Po zgłoszeniu przyjęcia dostawy w panelu pracownika punktu sprzedaży aplikacja przechodzi od razu do zgłaszania usterki (nie wiem, czy to celowo).
\end{itemize}

\end{document}