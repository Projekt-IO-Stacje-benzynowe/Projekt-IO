% Ten plik jest zaprojektowany do użycia z LuaLaTeX.

\documentclass[a4paper,12pt]{article}
\usepackage{fontspec} % Obsługa czcionek w LuaLaTeX
\usepackage[polish]{babel}
\usepackage{geometry} % Ustawienia marginesów
\geometry{margin=1in}
\usepackage{hyperref} % Obsługa hiperłączy
\usepackage{titlesec} % Formatowanie sekcji
\usepackage[skip=10pt plus1pt]{parskip}

% Ustawienia hiperłączy
\hypersetup{
    colorlinks=true,
    linkcolor=black,
    urlcolor=blue,
    pdftitle={Specyfikacja Wymagań Oprogramowania},
    pdfauthor={Diego Ostoja Kowalski}
}

% Tytuł dokumentu
\title{
    \line(1,0){250}\\
    System Zarządzania Promocjami\\
    Specyfikacja Wymagań Oprogramowania\\
    \line(1,0){250}}
\author{Antoni Blicharz\\
        Szymon Dybał\\
        Jakub Koszorz\\
        Mikołaj Mroczek\\
        Diego Ostoja-Kowalski\\}
\date{\today}

\begin{document}

\begin{titlepage}
    \maketitle
\end{titlepage}
\tableofcontents
\newpage

\section{Wprowadzenie}

\subsection{Cel dokumentu}
Celem tego dokumentu jest dostarczenie szczegółowego opisu wymagań dla Systemu Zarządzania Promocjami.
Dokument ten opisuje wymagania funkcjonalne i niefunkcjonalne, które muszą być spełnione przez produkt.
Jest on przeznaczony dla potencjalnych użytkowników oraz projektantów i programistów chcących wykorzystać ten system.

\subsection{Zakres produktu}
System ma na celu automatyzację procesu zarządzania promocjami produktów w sieciach handlowych takich jak stacje benzynowe, supermarkety i inne punkty sprzedaży detalicznej.
Umożliwi on pracownikom sieci handlowych dodawanie, edytowanie i usuwanie promocji, a także generowanie raportów dotyczących ich skuteczności.
System umożliwi też pracownikom koordynowanie dostaw nagród promocyjnych i utylizacji defektów.
Będzie on również umożliwiał zintegrowanie procesu wydawania produktów promocyjnych z systemem kasowym, co pozwoli na automatyczne zarządzanie stanami magazynowymi.


\subsection{Definicje, akronimy i skróty}

\textbf{Promocja} -- czas, w którym nagroda jest wydawana klientom po spełnieniu przez nich wymagań kupna konkretnych produktów podczas sprzedaży detalicznej.
Dopóki jeszcze nie nastąpiła data rozpoczęcia promocji, istnieje możliwość edycji wszystkich atrybutów promocji.
W trakcie jej trwania istnieje tylko możliwość edycji czasu zakończenia promocji, a po zakończeniu promocji nie ma możliwości zmiany jakichkolwiek jej atrybutów.

\textbf{Produkt} -- pojedynczy towar o określonym kodzie w systemie, który jest sprzedawany w punktach sprzedaży sieci handlowej.

\textbf{Nagroda} -- pojedynczy towar o określonym kodzie w systemie, który jest przekazywany klientowi po spełnieniu przez niego wymagań określonych w promocji.

\textbf{Punkt sprzedaży} -- pojedynczy punkt, w którym odbywa się sprzedaż detaliczna w ramach sieci handlowej.

\textbf{Dostawa} -- przekazanie określonej liczby nagród z magazynu centralnego do poszczególnych punktów sprzedaży.

\textbf{Magazyn/stan magazynowy} -- określa liczbę nagród o określonych kodach w określonym punkcie sprzedaży.

\textbf{Defekt} -- nagroda uszkodzona w trakcie transportu lub w trakcie przebywania w punkcie sprzedaży, nienadająca się do wydania klientowi.

\section{Opis ogólny}

\subsection{Perspektywa produktu}

System zarządzania promocjami jest zaprojektowany tak, aby był przystosowywalny jako część większego systemu zarządzania sprzedażą w sieciach handlowych.
Zawiera on interfejs do integracji z systemami kasowymi, jak również osobny system do zarządzania stanem magazynowym i generowania raportów.

\subsection{Funkcje produktu}

\begin{itemize}
    \item Tworzenie i edytowanie promocji
    \begin{itemize}
        \item Dodawanie nowej promocji
        \item Zmiana daty końcowej promocji
        \item Zmiana towarów promocyjnych przypisanych do promocji
        \item Usuwanie promocji
    \end{itemize}
    \item Integracja z systemem kasowym
    \begin{itemize}
        \item Obliczanie, jakie towary promocyjne należy wydać
        \item Automatyczne aktualizowanie stanów magazynowych
    \end{itemize}
    \item Zarządzanie stanem magazynowym
    \begin{itemize}
        \item Planowanie dostawy towarów promocyjnych
        \item Zgłaszanie przyjęcia dostawy towarów promocyjnych
        \item Monitorowanie stanów magazynowych
        \item Wysłanie zapytania o dodatkową dostawę towarów promocyjnych
        \item Wyświetlanie zapytań o dodatkowe dostawy towarów promocyjnych
    \end{itemize}
    \item Zarządzanie defektywnymi towarami promocyjnymi
    \begin{itemize}
        \item Zgłaszanie defektów towarów promocyjnych
        \item Planowanie utylizacji wadliwych towarów promocyjnych
    \end{itemize}
    \item Generowanie raportów dotyczących skuteczności promocji
\end{itemize}

\subsection{Klasy użytkowników}

Przewidywani użytkownicy Systemu Zarządzania Promocjami będą należeć do dokładnie jednej z poniższych pięciu grup:
\begin{itemize}
    \item \textbf{Administrator bazy danych} -- pracownik sieci handlowej zajmujący się utrzymaniem bazy danych przechowującej dane Systemu.
    Jest on doświadczony w używaniu języka SQL, utrzymywaniu baz danych systemów sieci handlowej i oczekuje łatwej do zrozumienia struktury tabel w dostarczonej bazie danych.
    \item \textbf{Analityk biznesowy} -- pracownik sieci handlowej odpowiedzialny za podsumowanie danych dotyczących promocji w celu podejmowania decyzji o przyszłych działaniach przedsiębiorstwa.
    Jest on wprawny w używaniu komputera osobistego i przyzwyczajony do używania wykresów do ilustracji danych.
    \item \textbf{Koordynator logistyki} -- pracownik sieci handlowej odpowiedzialny za planowanie dostaw nagród do punktów sprzedaży.
    Pracuje on na komputerze osobistym, ma doświadczenie w organizowaniu dostaw produktów sieci handlowej i wymaga klarownego interfejsu graficznego.
    \item \textbf{Koordynator promocji} -- pracownik sieci handlowej odpowiedzialny za planowanie nowych promocji i zarządzanie obecnymi.
    Pracuje on na komputerze osobistym, ma doświadczenie w organizowaniu promocji i wymaga klarownego interfejsu graficznego.
    \item \textbf{Pracownik punktu sprzedaży} -- pracownik sieci handlowej odpowiedzialny za przeprowadzanie transakcji sprzedaży detalicznej w punkcie sprzedaży.
    Pracuje on w jednym punkcie sprzedaży i wymaga przejrzystego, prostego interfejsu graficznego, którego da się używać z interfejsu systemu kasowego.
\end{itemize}

\subsection{Środowisko operacyjne}

System będzie działał na urządzeniach z systemem operacyjnym Windows 11 lub macOS 15 Sequoia.

\subsection{Ograniczenia implementacji}

System będzie ograniczony czasowo do połowy czerwca 2025, w czasie której musi zostać on dostarczony.
Ze względu na mały budżet zespołu deweloperskiego będzie on łączył się z pojedynczą bazą danych MySQL osadzoną na serwerach Amazon Web Services.
Przez łączenie się ze zdalną bazą danych za pośrednictwem Internetu, System będzie musiał implementować autoryzację użytkowników poprzez logowanie się w celu bezpieczeństwa danych sieci handlowej.

\subsection{Dokumentacja dla użytkownika}

Aplikacja będzie dostarczona wraz z instrukcją użytkownika dostępną w głównym folderze kodu źródłowego w pliku \texttt{docs/Instrukcja użytkownika.pdf}.

\section{Wymagania szczegółowe}

\subsection{Wymagania funkcjonalne}

\subsubsection{Zarządzanie promocjami}

Koordynatorzy promocji, którzy definiują i nadzorują proces planowania i przebiegu promocji, będą mieć dostęp do następujących funkcjonalności:
\begin{itemize}
    \item \textbf{Dodawanie promocji}:
    Promocja musi być określona przez daty rozpoczęcia i zakończenia, obejmowany produkt i oferowaną nagrodę, oraz informacje o ilości kupionych produktów, za którą jest wydawana nagroda.
    Promocja może być tylko planowana z datami rozpoczęcia i zakończenia późniejszymi od daty bieżącej, oraz datą zakończenia poźniejszą od daty rozpoczęcia. 
    \item \textbf{Edycja promocji}:
    Przed datą rozpoczęcia promocji można zmieniać dowolne dane określające promocję, po jej rozpoczęciu można zmieniać jedynie jej datę końcową, a po jej zakończeniu nie można zmieniać żadnych jej atrybutów.
    Edycja danych musi zachować warunki ,,data rozpoczęcia > data bieżąca'' i ,,data zakończenia > data rozpoczęcia''.
    Musi ona zawierać obejmowany produkt oraz wydawaną nagrodę.
    \item \textbf{Usuwanie promocji}:
    Promocja może zostać usunięta z systemu tylko wtedy, gdy jeszcze się nie rozpoczęła, t.j. musi spełniać warunek ,,data rozpoczęcia > data bieżąca''.
\end{itemize}

\subsubsection{Zarządzanie stanem magazynowym}

Koordynatorzy logistyki, którzy dbają, aby w trakcie trwania promocji nie następowało przepełnienie ani zbytnie opróżnienie magazynów nagród, będą mieć dostęp do następnych funkcjonalności:
\begin{itemize}
    \item \textbf{Monitorowanie stanu magazynowego}:
    Można wyświetlić bieżące stany magazynowe wszystkich punktów sprzedaży z podziałem na poszczególne nagrody.
    \item \textbf{Planowanie dostawy nagród}:
    Dostawa jest określana przez datę dostawy, poźniejszą od daty bieżącej, oraz ilości poszczególnych typów nagród dostarczanych w danej dostawie.
    Sam proces wysyłki nagród jest realizowany poza Systemem.
    \item \textbf{Wyświetlanie zapytań o dodatkowe dostawy nagród}:
    Dla każdego punktu sprzedaży, w którym zgłoszono zapytanie o dodatkową dostawę nagród, można wyświetlić wszystkie takie zapytania.
    Jeżeli zostanie ono uznane za zasadne, można na jego podstawie zaplanować nową dostawę nagród.
\end{itemize}

Pracownicy punktów sprzedaży, którzy zajmują się przyjmowaniem dostaw nagród i wydawaniem ich klientom sprzedaży detalicznej, będą mogli korzystać z poniższych funkcjonalności:
\begin{itemize}
    \item \textbf{Zgłaszanie przyjęcia dostawy nagród}:
    Po udanym procesie wysyłki nagród do punktu sprzedaży, jego pracownik musi zatwierdzić jego przyjęcie w Systemie.
    Tylko w ten sposób może on zwiększyć stan magazynowy swojego punktu sprzedaży.
    W trakcie zgłaszania przyjęcia dostawy można od razu zgłosić liczbę defektów w dostawie.
    \item \textbf{Wysłanie zapytania o dodatkową dostawę nagród}:
    Pracownik punktu sprzedaży może samodzielnie kontrolować stan magazynowy swojego punktu sprzedaży i wysłać zapytanie o dodatkową dostawę nagród, jeżeli jest ich za mało w magazynie.
    Takie zapytanie jest potem możliwa do wyświetlenia przez koordynatora logistyki.
\end{itemize}

\subsubsection{Integracja z systemem kasowym}

System kasowy w punkcie sprzedaży, oprócz już w nim istniejącej funkcjonalności obsługi transakcji detalicznych, będzie również sprawdzał, jaka ilość poszczególnych nagród należy się klientowi sprzedaży detalicznej.
Będzie on wywoływał funkcję przez API Systemu Zarządzania Promocjami, w którym jest zalogowany pracownik punktu sprzedaży, która przyjmie liczby produktów sprzedanych w obecnej transakcji i odpowie informacjami o należących się nagrodach.
Po potwierdzeniu wydania ich przez pracownika nastąpi odjęcie za stanu magazynowego odpowiedniej liczby wydanych nagród.

\subsubsection{Zarządzanie defektami}

Pracownik punktu sprzedaży może zgłosić defekty, które powstały podczas transportu lub podczas przebywania w punkcie sprzedaży.
Po zgłoszeniu określonej liczby defektów danego typu nagrody System odejmuje od stanu magazynowego tej nagrody podaną liczbę sztuk, jeżeli nie przekracza ona obecnego stanu magazynowego.

\subsubsection{Generowanie raportów}

Analityk biznesowy, który zajmuje się analizą efektów promocji na sprzedaż produktów, ma możliwość generowania następujących raportów w formacie PDF:
\begin{itemize}
    \item \textbf{Raport produktu}:
    Ilustruje on na wykresach słupkowych liczbę sprzedaży danego produktu w jednym roku w poszczególnych miesiącach.
    Może on być generowany dla całej sieci handlowej lub dla poszczególnych punktów sprzedaży.
    \item \textbf{Raport nagród}:
    Ilustruje on na wykresach słupkowych liczbę wydań nagród w jednym roku w poszczególnych miesiącach.
    Może on być generowany dla całej sieci handlowej lub dla poszczególnych punktów sprzedaży.
\end{itemize}

\subsubsection{Dodatkowe funkcjonalności}

Administrator bazy danych, który zajmuje się utrzymaniem bazy danych Systemu, ma dostęp do następujących funkcjonalności:
\begin{itemize}
    \item \textbf{Zarządzanie użytkownikami}:
    Administrator bazy danych definiuje użytkowników Systemu przy użyciu bazy danych.
    Każdy użytkownik posiada unikalny identyfikator, adres email, hasło oraz jest przypisany do odpowiedniej klasy użytkowników.
    \item \textbf{Zarządzanie nagrodami}:
    Administrator bazy danych definiuje nagrody przy użyciu bazy danych.
    Każda nagroda posiada unikalny identyfikator, nazwę oraz koszt jednostkowy jej zakupu od producenta.
    \item \textbf{Zarządzanie produktami}:
    Administrator bazy danych definiuje produkty przy użyciu bazy danych.
    Każdy produkt posiada unikalny identyfikator, nazwę, koszt jednostkowy jego zakupu od producenta i cenę sprzedaży detalicznej.
    \item \textbf{Zarządzanie punktami sprzedaży}:
    Administrator bazy danych definiuje punkty sprzedaży przy użyciu bazy danych.
    Każdy punkt sprzedaży posiada unikalny identyfikator, nazwę i adres.
\end{itemize}

\subsection{Wymagania interfejsów}

\subsubsection{Interfejsy użytkownika}

Wszyscy użytkownicy Systemu Zarządzania Promocjami poza administratorami bazy danych potrzebują czytelnych ekranów do realizacji wszystkich swoich funkcjonalności.
Pierwszym jego widocznym ekranem jest ekran logowania, w którym użytkownik podaje swój adres email i hasło, które są autentyfikowane przez bazę danych.
Po zalogowaniu się do Systemu użytkownik widzi odpowiedni ekran, zgodny z jego klasą, na którym umieszczone są przyciski.
Wybór przycisku powoduje przejście do odpowiedniej akcji, szczegóły tego procesu są opisane w pliku \texttt{docs/Diagram czynności.pdf}.

\subsubsection{Interfejsy programowe}

Wszystkie funkcjonalności powinny być wykonywane poprzez połączenie internetowe ze zdalną bazą danych Systemu, które wymaga logowania użytkownika.

\subsubsection{Interfejsy sprzętowe}

Nie zidentyfikowano żadnych wymaganych interfejsów sprzętowych.

\subsubsection{Interfejsy komunikacyjne}

Funkcjonalność integracji z systemem kasowym powinna być realizowana przez API dostępne lokalnie na urządzeniu, na którym jednocześnie działa system kasowy oraz interfejs użytkownika, do którego zalogowany jest pracownik punktu sprzedaży. 

\subsection{Wymagania niefunkcjonalne}

\subsubsection{Użyteczność}

Cały interfejs użytkownika ma być realizowany w języku angielskim i mieć kolorystykę umożliwiającą korzystanie z niego przez osoby ze ślepotą barw.
System musi posiadać mechanizm logowania, który daje użytkownikom dostęp do funkcjonalności odpowiadających ich klasie użytkownika.

\subsubsection{Bezpieczeństwo}

System powinien udostępniać użytkownikom dostęp tylko do czytania, tworzenia, modyfikowania i usuwania tylko tych rekordów bazy danych, z którymi należy wykonywać te czynności, aby wykonywać funkcjonalności wymagane dla danego użytkownika.

\subsubsection{Wydajność}

System powinien obsługiwać do 1000 punktów sprzedaży i w zakresie jednoczesnej aktywności do 20\% wszystkich punktów sprzedaży powinien zwracać odpowiedź w ciągu maksymalnie 10 sekund.

\subsubsection{Przenaszalność}

Interfejs użytkownika powinien działać na urządzeniach z systemem Windows 11 lub macOS 15 Sequoia, gdzie jest zainstalowane oprogramowanie Java SE Development Kit w wersji 24.0.1 oraz JavaFX w wersji 24.0.1.
Baza danych powinna działać na serwerze z zainstalowanym oprogramowaniem MySQL w wersji 8.0.33.

\section{Załączniki}

\begin{itemize}
    \item Załącznik 1: ...
    \item Załącznik 2: ...
\end{itemize}

\end{document}