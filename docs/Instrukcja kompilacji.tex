% Ten plik jest zaprojektowany do użycia z LuaLaTeX.

\documentclass[a4paper,12pt]{article}
\usepackage{fontspec} % Obsługa czcionek w LuaLaTeX
\usepackage[polish]{babel}
\usepackage{geometry} % Ustawienia marginesów
\geometry{margin=1in}
\usepackage{hyperref} % Obsługa hiperłączy
\usepackage{titlesec} % Formatowanie sekcji
\usepackage[skip=10pt plus1pt]{parskip}
\usepackage[table]{xcolor}
\usepackage{float}

\usepackage{xcolor}

\usepackage{url}

% Ustawienia hiperłączy
\hypersetup{
    colorlinks=true,
    linkcolor=black,
    urlcolor=black,
    pdftitle={Instrukcja kompilacji},
    pdfauthor={Diego Ostoja Kowalski}
}

% Tytuł dokumentu
\title{
    \line(1,0){250}\\
    System Zarządzania Promocjami\\
    Instrukcja kompilacji\\
    \line(1,0){250}}
\author{Antoni Blicharz\\
        Szymon Dybał\\
        Jakub Koszorz\\
        Mikołaj Mroczek\\
        Diego Ostoja-Kowalski\\}
\date{\today}

\begin{document}

\begin{titlepage}
    \maketitle
\end{titlepage}

\newpage

\section{Kompilacja od podstaw}

Pliki \texttt{cleanInstall.bat} oraz \texttt{startIO.bat} służą do kompilacji programu Java wprost z kodu, bez potrzeby ręcznego instalowania JDK, JavaFX, Maven oraz jego pluginów.
Nadal jest jednak wymagane uprzednie zainstalowanie Java SE w wersji 24.0.1.
W wierszu poleceń trzeba przejść do poziomu folderu, w którym są umieszczone wszystkie pliki projektu, a następnie uruchomić te dwa pliki w podanej kolejności (np. w Windows Command Line wystarczy wywołać polecenia \texttt{\~{}/cleanInstall.bat} oraz \texttt{\~{}/startIO.bat}).
Następnie program zostanie uruchomiony z pliku \texttt{target/app-java-1.0-SNAPSHOT.jar}, którego można potem używać.

\section{Kompilacja bez używania skryptów}

Do tej kompilacji potrzebne są zainstalowane następujące narzędzia:
\begin{itemize}
    \item Java Development Kit w wersji 24.0.1 z ustawioną ścieżką w zmiennej środowiskowej \texttt{PATH}.
    \item JavaFX w wersji 24.0.1 z ustawioną ścieżką w zmiennej środowiskowej \texttt{PATH}.
    \item Maven w wersji 4.0.0.
\end{itemize}
Jest również potrzebny plik \texttt{pom.xml}, aby zostały zainstalowane i użyte odpowiednie pluginy Maven.

Kompilację następnie wykonuje się przez użycie w wierszu poleceń w folderze z plikami projektu kolejno dwóch poleceń: najpierw \texttt{\~{}/mvn clean install}, a następnie \texttt{\~{}/mvn javafx:run}.
Takie rozwiązanie od razu również uruchomi aplikację, aby tylko ją skompilować, należy drugie polecenie zastąpić przez polecenie \texttt{\~{}/mvn package}.
Wynik kompilacji to plik \texttt{target/app-java-1.0-SNAPSHOT.jar}.

\end{document}