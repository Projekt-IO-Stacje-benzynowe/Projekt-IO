% Ten plik jest zaprojektowany do użycia z LuaLaTeX.

\documentclass[a4paper,12pt]{article}
\usepackage{fontspec} % Obsługa czcionek w LuaLaTeX
\usepackage[polish]{babel}
\usepackage{geometry} % Ustawienia marginesów
\geometry{margin=1in}
\usepackage{hyperref} % Obsługa hiperłączy
\usepackage{titlesec} % Formatowanie sekcji
\usepackage[skip=10pt plus1pt]{parskip}
\usepackage[table]{xcolor}
\usepackage{float}

\usepackage{xcolor}

\usepackage{url}

% Ustawienia hiperłączy
\hypersetup{
    colorlinks=true,
    linkcolor=black,
    urlcolor=black,
    pdftitle={Opis oprogramowania},
    pdfauthor={Diego Ostoja Kowalski}
}

% Tytuł dokumentu
\title{
    \line(1,0){250}\\
    System Zarządzania Promocjami\\
    Opis oprogramowania\\
    \line(1,0){250}}
\author{Antoni Blicharz\\
        Szymon Dybał\\
        Jakub Koszorz\\
        Mikołaj Mroczek\\
        Diego Ostoja-Kowalski\\}
\date{\today}

\begin{document}

\begin{titlepage}
    \maketitle
\end{titlepage}

\newpage

\section{Wstęp}

\subsection{Cel dokumentu}

Ten dokument opisuje architekturę oprogramowania Systemu Zarządzania Promocjami.
Tłumaczy on zastosowany wzorzec architekturalny oraz to, które części oprogramowania realizują którą jego część.
Opisuje on również, dlaczego zostały zastosowane przez nas różne narzędzia do stworzenia oprogramowania.

\subsection{Definicje}

\textbf{Panel} -- wyświetlana wersja intefejsu graficznego.
Panele mogą zawierać inne panele oraz być przełączane na inne panele realizujące inne funkcjonalności.

Dla definicji innych terminów zobacz sekcję ,,Definicje'' w dokumencie\linebreak\texttt{Software Requirements Specification.pdf}.

\subsection{Szczegóły implementacji}

Szczegóły implementacji Systemu można znaleźć w folderze \texttt{src} oraz dokumencie \texttt{Diagram klas.png}, gdzie znajdują się informacje na temat kodu napisanego w języku Java, oraz w pliku \texttt{Diagram bazy danych.png}, gdzie jest przedstawiona struktura zdalnej bazy danych.

\section{Wzorzec MVCS}

Podczas gdy bardziej popularny wzorzec MVC (Model-View-Controller) ma jeden ustalony kształt, taka zgoda nie istnieje co do jego modyfikacji MVCS (Model-View-Controller-Service).
Opierając się na idei takiej jak \href{https://quantiphi.com/blog/an-introduction-to-mvcs-architecture/}{\textcolor{blue}{przedstawiona tutaj}}, w naszej architekturze cztery warstwy wzorca MVCS pełnią następujące funkcje:
\begin{itemize}
    \item \textbf{Model} -- warstwa zawierająca encje reprezentujące dane Systemu.
    Zaliczają się do tego wszystkie klasy \texttt{*Model.java} zdefiniowane w \url{src/main/java/app/model}, jak również struktura bazy danych.
    Klasy służą reprezentacji tych danych w interfejsie graficznym, a baza danych przechowuje końcowy stan operacji użytkownika.
    \item \textbf{Service} -- warstwa służąca realizacji logiki biznesowej.
    Zaliczają się do niej klasy zdefiniowane w \url{src/main/java/app/service} poza \texttt{SceneManager.java}.
    Służy ona dwustronnej wymianie danych pomiędzy bazą danych, a interfejsem graficznym użytkownika, to w niej następuje sprawdzanie poprawności danych i wywoływanie powiadomień o powodzeniu lub niepowodzeniu operacji wykonywanych przez użytkowników.
    \item \textbf{Controller} -- warstwa służąca transformacji danych przekazanych z warstwy Service na potrzeby interfejsu graficznego oraz przekazywaniu danych wprowadzonych przez użytkownika do warstwy Service.
    Należą do niej klasy zdefiniowane w \texttt{src/main/java/app/controllers}.
    \item \textbf{View} -- warstwa odpowiadająca za faktyczne wyświetlanie interfejsu graficznego, komunikująca się z warstwą \texttt{Controller}.
    Należą do niej panele \texttt{.fxml} zawarte w \texttt{src/main/java/app/resources}, jak również klasy \texttt{PanelList.java} i\linebreak\texttt{SceneManager.java}.
\end{itemize}

Została ona wybrana ze względu na prostotę jej implementacji z użyciem JavaFX oraz jasny podział obowiązków warstw.
Definiowanie panelu w pliku \texttt{.fxml} jest bardzo szybkie i łatwo jest utworzyć nowy panel z odpowiednimi elementami, a następnie zdefiniować odpowiednie metody klas w warstwach Controller i Service, rozdzielając między nie niezbędne importy klas takich jak \texttt{RepositorySQL.java} czy \texttt{SceneManager.java} do nich i ograniczając skomplikowanie zależności modułów.

Większość plików oprogramowania w prosty sposób realizuje tę architekturę przez powiązanie ze sobą odpowiadających sobie plików \texttt{*Service.java}, \texttt{*Controller.java} oraz \url{*\textunderscore panel.fxml}, gdzie pierwsze dwa pliki są nazwane w CamelCase, a ostatni w snake\textunderscore{}case.
Pliki \texttt{*Model.java} są z kolei używane uniwersalnie, w zależności od funkcjonalności wymaganych w konkretnych panelach.
Pozostałe pliki, które nie zaliczają się do tego schematu, są opisane dokładniej w dalszej sekcji.

\section{Opis szczegółowy}

\subsection{Baza danych}

Klasa \texttt{MySQLConnection.java} jest odpowiedzialna za nawiązywanie i kończenie połączenia z bazą danych MySQL.
Klasa \texttt{RepositorySQL.java} jest odpowiedzialna za wykonywanie odpowiednich poleceń SQL w celu wyświetlenia obecnych danych lub modyfikacji bazy danych.

\subsection{Połączenie z systemem kasowym}

Klasy odpowiedzialne za realizację tego połączenia oraz symulację takowego połączenia na potrzeby wykazania funkcjonalności znajdują się w \url{src/main/java/app/service/branch\textunderscore panel/ClientSimulation}.

\subsection{Funkcjonalności pracownika punktu sprzedaży}

Część klas warstwy Service obsługująca pracownika punktu sprzedaży została połączona, gdyż następowała zbędna duplikacja kodu.

\subsection{Tworzenie raportów}

Istnieją dodatkowe klasy służące tworzeniu raportów \texttt{.pdf}, zawarte w \url{src/main/java/app/service/business\textunderscore panel}.
Służą one używaniu tego samego kodu do generacji podobnych graficznie raportów zamiast powielania logiki w różnych panelach odpowiadających wyborowi różnych raportów.

\subsection{Sprawdzanie poprawności danych i wyświetlanie powiadomień}

Klasy \texttt{LoginValidation.java} i \texttt{TypeValidation.java} służą sprawdzaniu poprawności wprowadzanych danych, a klasy \texttt{Alerts.java} i \texttt{Dialogs.java} zawierają kilkukrotnie używane typy okienek z powiadomieniami oraz okienek dialogowych.

\subsection{Wyświetlanie interfejsu graficznego}

Klasa \texttt{Main.java} jest odpowiedzialna za rozpoczęcie działania interfejsu graficznego.
Klasa \texttt{SceneManager.java} służy ładowaniu, usuwaniu i przełączaniu między panelami, wykorzystując do tego mapę ich nazw w klasie \texttt{PanelList.java}.
Klasy zawarte w \url{src/main/java/app/controllers/shared} oraz powiązane z nimi pliki \texttt{.fxml} służą wyświetlaniu bocznego panelu użytkownika i zapewnianiu miejsca na podpanele realizujące wymagania funkcjonalne.

\subsection{Przechowywanie wybranych opcji}

Klasa \texttt{Session.java} jest odpowiedzialna za przechowywanie informacji o zalogowanym użytkowniku oraz o elementach wybranych w poprzednich panelach t.j. punkt sprzedaży czy planowana dostawa do modyfikacji.

\subsection{Tworzenie tabel}

Ze względu na większą złożoność procesu tworzenia tabel, to zadanie zostało oddelegowane do specjalnej klasy \texttt{TableService.java}.

\section{Zastosowane narzędzia}

W projekcie zastosowano następujące narzędzia:
\begin{itemize}
    \item \textbf{Java} -- zespół miał już wcześniej styczność z tym językiem programowania oraz wczesne eksploracje wskazały na bogatą gamę narzędzi do tworzenia aplikacji okienkowych i łączność z bazą danych, co dokładnie odpowiadało potrzebom projektu.
    \item \textbf{JavaFX} -- ten zestaw rozszerzeń do języka Java okazał się być zasobny w przeróżne gotowe elementy interfejsu graficznego, co uczyniło proces deweloperski bardzo prosty.
    Zwłaszcza przydatne były takie elementy jak \texttt{TableView} i \texttt{ComboBox}, które można od razu populować własnymi obiektami warstwy Model i które automatycznie rejestrują i przekazują informację o zaznaczonym polu, co było niesamowitym uproszczeniem logiki wyświetlania i pobierania z nich danych.
    Podobieństwo plików \texttt{.fxml} do struktury języka HTML ułatwiła też wdrożenie się części zespołu zaznajomionej z tworzeniem stron internetowych.
    \item \textbf{MySQL} -- odmiana SQL, która szybko działa na serwerach zdalnych usługi AWS, jednocześnie naliczając znacznie mniej kosztów, niż np. Microsoft SQL Server.
    \item \textbf{Maven} -- narzędzie umożliwiające usprawnienie procesu kompilacji i dodawania nowych modułów (widocznych w pliku \texttt{pom.xml}), co pozwala na korzystanie z gotowych rozwiązań bardziej zaawansowanych problemów i większą przenaszalność kodu dla zespołu deweloperskiego.
    Kluczowe były tutaj pluginy do włączania JavaFX (\texttt{javafx-controls}, \texttt{javafx-fxml}), łączenia się z bazą danych (\texttt{mysql-connector-java}), tworzenia plików PDF i wypełniania ich wykresami (\texttt{pdfbox}, \texttt{itextpdf}, \texttt{jfreechart}, \texttt{fontbox}) oraz utworzenia API dla systemu kasowego komunikującego się plikami JSON (\texttt{jackson-databind}).
\end{itemize}

\end{document}